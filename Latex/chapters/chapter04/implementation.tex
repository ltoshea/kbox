%
% File: chap01.tex
% Author: Liam O'Shea
% Description: Introduction chapter where the boxing goes.
%
\let\textcircled=\pgftextcircled
\chapter{Implementation}
\label{chap:intro}

\initial{B}egins a chapter. Example: When the beloved cellist (Christopher Walken - outstanding) of a world-renowned string quartet receives a life-changing diagnosis, the group's future suddenly hangs in the balance: suppressed emotions, competing egos and uncontrollable passions threaten to derail years of friendship and collaboration. Featuring a brilliant ensemble cast (including Philip Seymour Hoffman, Catherine Keener and Mark Ivanir as the three other quartet members), it is a fascinating look into the world of working musicians, and an elegant homage to chamber music and the cultural world of New York. The music, of course, is ravishing (the score is the work of regular David Lynch collaborator Angelo Badalamenti): A Late Quartet hits all the right notes.

%=======
\section{Comparison Methods}
\label{sec:sec01}

Hidden Markov models?

\subsection{PCA}
\label{subsec:subsec01}

Principal Component Analysis is a statistical procedure that transformed a set of observations of
potentially correlated variables into a set of linearly uncorrelated variables called principal components. The number of principal components should always be less than or equal to the number of original values with the first principal component having the largest possible variance.
In my case I will be looking to reduce my 20 points per frame for each joint into a low dimensionality set that will help me to uniquely identify punches.

Svm?\newline
Chapter 3: Specification \& Design\newline
Scope Algorithms\newline
Chapter 4: Implementation\newline
Chapter 5: Data capture??\newline
Need to make and collect data consent forms to run a study?\newline
Need to gather more data?\newline

\paragraph{Data Format}
I record data from the Kinect in a space separated text file with each line corresponding to one timeframe. The structure of a line is: 
tracking_flag x_0 y_0 z_0 tracking_flag x_1 y_1 z_1 ... tracking_flag x_19 y_19 z_19,
where x_i,y_i,z_i are the x,y,z coordinates representing the position of the ith joint.
Each new line is represented by a very large value that could not represent a Kinect measurement. (e.g. 2000000) 
The tracking_flag is an integer which describes the status of the joint:
Joint not tracked = 0, Joint position inferred = 1, Join position tracked = 2.
If the joint is not tracked the position is set to (-10000, -10000, -10000) and it should not be used.
The position of the camera is (0,0,0).


The joints are:
i=0: NUI_SKELETON_POSITION_HIP_CENTER
i=1: NUI_SKELETON_POSITION_SPINE
i=2: NUI_SKELETON_POSITION_SHOULDER_CENTER
i=3: NUI_SKELETON_POSITION_HEAD
i=4: NUI_SKELETON_POSITION_SHOULDER_LEFT
i=5: NUI_SKELETON_POSITION_ELBOW_LEFT
i=6: NUI_SKELETON_POSITION_WRIST_LEFT
i=7: NUI_SKELETON_POSITION_HAND_LEFT
i=8: NUI_SKELETON_POSITION_SHOULDER_RIGHT
i=9: NUI_SKELETON_POSITION_ELBOW_RIGHT
i=10: NUI_SKELETON_POSITION_WRIST_RIGHT
i=11: NUI_SKELETON_POSITION_HAND_RIGHT
i=12: NUI_SKELETON_POSITION_HIP_LEFT
i=13: NUI_SKELETON_POSITION_KNEE_LEFT
i=14: NUI_SKELETON_POSITION_ANKLE_LEFT
i=15: NUI_SKELETON_POSITION_FOOT_LEFT
i=16: NUI_SKELETON_POSITION_HIP_RIGHT
i=17: NUI_SKELETON_POSITION_KNEE_RIGHT
i=18: NUI_SKELETON_POSITION_ANKLE_RIGHT
i=19: NUI_SKELETON_POSITION_FOOT_RIGHT

%=========================================================